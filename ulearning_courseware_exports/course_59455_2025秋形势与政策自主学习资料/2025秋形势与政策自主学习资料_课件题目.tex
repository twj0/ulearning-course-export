\documentclass[12pt,UTF8]{ctexart}
\usepackage{graphicx}
\usepackage{amsmath, amsfonts, amssymb}
\usepackage[a4paper, margin=1in]{geometry}
\usepackage{enumitem}
\usepackage{hyperref}
\hypersetup{colorlinks=true, linkcolor=blue, urlcolor=blue, citecolor=green}
\usepackage{array,longtable}
\usepackage{float}
\title{2025秋形势与政策自主学习资料 - 课件题目汇总}
\author{优学院导出}
\date{2025-09-12}
\begin{document}
\maketitle


\section*{中国新型大国关系构建专题}
\hrulefill

\subsection*{页面一}

\subsubsection*{8. (单选题) \small QID: 19302791}

\textbf{题干:}
()是最早与新中国建立正式外交关系的西方大国。



\textbf{选项:}
\begin{itemize}[leftmargin=*]

  \item A. 美国

  \item B. 法国

  \item C. 英国

  \item D. 德国

\end{itemize}

\textbf{正确答案:}
B

\vspace{0.3em}\hrulefill\vspace{0.7em}

\subsection*{页面一}

\subsubsection*{9. (多选题) \small QID: 19302792}

\textbf{题干:}
习近平总书记倡导的全人类共同价值包括以下哪些内容?(  )



\textbf{选项:}
\begin{itemize}[leftmargin=*]

  \item A. 和平

  \item B. 发展

  \item C. 公平

  \item D. 正义

  \item E. 民主

  \item F. 自由

\end{itemize}

\textbf{正确答案:}
A | B | C | D | E | F

\vspace{0.3em}\hrulefill\vspace{0.7em}

\subsection*{一、习近平外交思想关于大国关系的战略思维和理论创新}

\subsubsection*{14. (单选题) \small QID: 19302800}

\textbf{题干:}
中巴(西)之间的双边关系被定义为?



\textbf{选项:}
\begin{itemize}[leftmargin=*]

  \item A. 普通合作伙伴

  \item B. 全面战略伙伴

  \item C. 区域性经济盟友

  \item D. 军事同盟

\end{itemize}

\textbf{正确答案:}
B

\vspace{0.3em}\hrulefill\vspace{0.7em}

\subsection*{二、习近平外交思想指导我们看清大国关系的本质}

\subsubsection*{17. (单选题) \small QID: 19302804}

\textbf{题干:}
()是提升大国关系正能量和削弱负能量的主要杠杆和途径。



\textbf{选项:}
\begin{itemize}[leftmargin=*]

  \item A. 军事扩张

  \item B. 经济科技发展

  \item C. 意识形态输出

  \item D. 单边制裁

\end{itemize}

\textbf{正确答案:}
B

\vspace{0.3em}\hrulefill\vspace{0.7em}

\subsection*{三、中国努力构建大国关系新格局}

\subsubsection*{28. (单选题) \small QID: 19302815}

\textbf{题干:}
()成为中国周边首个战略伙伴集群。



\textbf{选项:}
\begin{itemize}[leftmargin=*]

  \item A. 东盟

  \item B. 南亚

  \item C. 中亚

  \item D. 东北亚

\end{itemize}

\textbf{正确答案:}
C

\vspace{0.3em}\hrulefill\vspace{0.7em}

\subsection*{页面一}

\subsubsection*{39. (单选题) \small QID: 19302826}

\textbf{题干:}
2013—2024年,中欧班列已覆盖亚欧多少个国家的300多个城市?



\textbf{选项:}
\begin{itemize}[leftmargin=*]

  \item A. 20国

  \item B. 28国

  \item C. 36国

  \item D. 50国

\end{itemize}

\textbf{正确答案:}
C

\vspace{0.3em}\hrulefill\vspace{0.7em}

\subsection*{页面一}

\subsubsection*{50. (多选题) \small QID: 19302837}

\textbf{题干:}
中国以建设()为共同愿景积极打造周边命运共同体。



\textbf{选项:}
\begin{itemize}[leftmargin=*]

  \item A. 和平

  \item B. 安宁

  \item C. 繁荣

  \item D. 美丽

  \item E. 友好

\end{itemize}

\textbf{正确答案:}
A | B | C | D | E

\vspace{0.3em}\hrulefill\vspace{0.7em}

\section*{新时代中国周边外交专题}
\hrulefill

\subsection*{页面一}

\subsubsection*{1. (单选题) \small QID: 19302838}

\textbf{题干:}
习近平总书记正式提出周边命运共同体概念的时间是()?



\textbf{选项:}
\begin{itemize}[leftmargin=*]

  \item A. 2012

  \item B. 2013

  \item C. 2014

  \item D. 2022

\end{itemize}

\textbf{正确答案:}
C

\vspace{0.3em}\hrulefill\vspace{0.7em}

\subsection*{(一)中国与周边国家的政治外交关系更为稳固}

\subsubsection*{3. (单选题) \small QID: 19302841}

\textbf{题干:}
2012年至2020年间,中国与周边国家元首互访达到 ()次,平均每年高达 31 次,其中周边国家赴中国进行的元首访问为 ()次,占到互访总数的 75\%。



\textbf{选项:}
\begin{itemize}[leftmargin=*]

  \item A. 218;163

  \item B. 200;170

  \item C. 14;18

  \item D. 219;165

\end{itemize}

\textbf{正确答案:}
A

\vspace{0.3em}\hrulefill\vspace{0.7em}

\subsection*{(二)中国与周边国家的经济纽带更为紧密}

\subsubsection*{9. (单选题) \small QID: 19302848}

\textbf{题干:}
中国投资兴建的全长1035公里的()于 2021年12 月全线正式通车,从此老挝一下子进入动车时代,变成了“陆联国”。



\textbf{选项:}
\begin{itemize}[leftmargin=*]

  \item A. 中老铁路

  \item B. 雅万高铁

  \item C. 中印高铁

  \item D. 印老铁路

\end{itemize}

\textbf{正确答案:}
A

\vspace{0.3em}\hrulefill\vspace{0.7em}

\subsection*{(三)中国与周边国家的人文交流更为密切}

\subsubsection*{10. (单选题) \small QID: 19302849}

\textbf{题干:}
2014 年,中哈吉跨国联合申报()在第 38 届世界遗产大会上被列入《世界遗产名录》。



\textbf{选项:}
\begin{itemize}[leftmargin=*]

  \item A. 《河内,河内》

  \item B. 《舞乐传奇》

  \item C. 《圣地曙光》

  \item D. “丝绸之路:长安—天山廊道的路网”

\end{itemize}

\textbf{正确答案:}
D

\vspace{0.3em}\hrulefill\vspace{0.7em}

\subsection*{页面一}

\subsubsection*{20. (单选题) \small QID: 19302861}

\textbf{题干:}
()已经成为推动中国与周边国家经济合作的重要平台,周边地区也成为中国推动区域多边合作的集聚区。



\textbf{选项:}
\begin{itemize}[leftmargin=*]

  \item A. 上海合作组织

  \item B. 区域多边合作机制

  \item C. “10+1”领导人会议

  \item D. 中日韩领导人会议

\end{itemize}

\textbf{正确答案:}
B

\vspace{0.3em}\hrulefill\vspace{0.7em}

\subsection*{概要}

\subsubsection*{21. (多选题) \small QID: 19302862}

\textbf{题干:}
2025年中央周边工作会议后,习近平主席年内首访选择了()、()和(),彰显了东南亚在中国周边外交中的独特地位。



\textbf{选项:}
\begin{itemize}[leftmargin=*]

  \item A. 老挝

  \item B. 越南

  \item C. 马来西亚

  \item D. 柬埔寨

\end{itemize}

\textbf{正确答案:}
B | C | D

\vspace{0.3em}\hrulefill\vspace{0.7em}

\subsection*{(一)}

\subsubsection*{22. (单选题) \small QID: 19302863}

\textbf{题干:}
据东南亚民调显示,()连续多年被东盟视为最具政治和战略影响力的国家。



\textbf{选项:}
\begin{itemize}[leftmargin=*]

  \item A. 中国

  \item B. 越南

  \item C. 印度

  \item D. 美国

\end{itemize}

\textbf{正确答案:}
A

\vspace{0.3em}\hrulefill\vspace{0.7em}

\subsection*{(三)}

\subsubsection*{24. (单选题) \small QID: 19302866}

\textbf{题干:}
特朗普上台后,肆意挥舞关税大棒,冲击全球贸易秩序。()成为美国“对等关税”的重灾区,中南半岛国家受冲击尤甚。



\textbf{选项:}
\begin{itemize}[leftmargin=*]

  \item A. 朝鲜半岛

  \item B. 东南亚

  \item C. 日韩

  \item D. 东亚

\end{itemize}

\textbf{正确答案:}
B

\vspace{0.3em}\hrulefill\vspace{0.7em}

\subsection*{(四)}

\subsubsection*{26. (多选题) \small QID: 19302868}

\textbf{题干:}
地区国家对美国愈发“不托底”的心理,已经一定程度改变其“()、()”的思维定势和路径依赖。



\textbf{选项:}
\begin{itemize}[leftmargin=*]

  \item A. 经济倚美

  \item B. 经济倚华

  \item C. 安全靠美

  \item D. 安全靠中

\end{itemize}

\textbf{正确答案:}
B | C

\vspace{0.3em}\hrulefill\vspace{0.7em}

\subsection*{(五)}

\subsubsection*{28. (单选题) \small QID: 19302870}

\textbf{题干:}
()是破解阻碍周边地区经济发展结构性难题的重要抓手,对地区各国的经济复苏与经济发展都至关重要。



\textbf{选项:}
\begin{itemize}[leftmargin=*]

  \item A. 周边区域合作

  \item B. 周边政治互信

  \item C. 周边人文交流

  \item D. 周边合作竞争

\end{itemize}

\textbf{正确答案:}
A

\vspace{0.3em}\hrulefill\vspace{0.7em}

\subsection*{(一)被殖民压迫和反帝反殖民的共同记忆}

\subsubsection*{30. (单选题) \small QID: 19302974}

\textbf{题干:}
孙中山在 1900—1908 年间多次赴越南、新加坡等地,以()为革命基地建立同盟会分会,并得到了包括华人华侨在内的东南亚革命力量的支持,而他领导的辛亥革命成为东南亚反殖民斗争的重要思想源泉。



\textbf{选项:}
\begin{itemize}[leftmargin=*]

  \item A. 日本

  \item B. 美国

  \item C. 东南亚

  \item D. 法国

\end{itemize}

\textbf{正确答案:}
C

\vspace{0.3em}\hrulefill\vspace{0.7em}

\subsection*{(二)共同超越冷战“零和博弈”的政治理念}

\subsubsection*{32. (多选题) \small QID: 19302976}

\textbf{题干:}
万隆会议提出的处理国家间关系的十项原则,孕育出以“()、()、()”为内涵的“万隆精神”。



\textbf{选项:}
\begin{itemize}[leftmargin=*]

  \item A. 团结

  \item B. 合作

  \item C. 竞争

  \item D. 共同发展

\end{itemize}

\textbf{正确答案:}
A | B | D

\vspace{0.3em}\hrulefill\vspace{0.7em}

\subsection*{(三)共同危机下“南南合作”的互助记忆}

\subsubsection*{36. (多选题) \small QID: 19302981}

\textbf{题干:}
中国和东盟国家通过“一带一路”倡议与《东盟互联互通总体规划2025》的战略对接,正在继续创新南南合作模式的发展。这种模式充分尊重发起国彼此的主体性,是一种“()、()、()”的新型合作框架。



\textbf{选项:}
\begin{itemize}[leftmargin=*]

  \item A. 共商

  \item B. 共建

  \item C. 共享

  \item D. 竞争

\end{itemize}

\textbf{正确答案:}
A | B | C

\vspace{0.3em}\hrulefill\vspace{0.7em}

\subsection*{(一)双向奔赴的经济合作新格局}

\subsubsection*{37. (单选题) \small QID: 19302982}

\textbf{题干:}
在软联通方面,双方签署《中国-东盟关于推动建立可持续和包容性的数字生态合作联合声明》,中国-东盟自贸区 3.0 版首次将()专章纳入其中。



\textbf{选项:}
\begin{itemize}[leftmargin=*]

  \item A. 工业经济

  \item B. 铁路经济

  \item C. 数字经济

  \item D. 交通发展

\end{itemize}

\textbf{正确答案:}
C

\vspace{0.3em}\hrulefill\vspace{0.7em}

\subsection*{(二)共同塑造的亚洲安全新模式}

\subsubsection*{41. (单选题) \small QID: 19302986}

\textbf{题干:}
中国与东盟国家在湄公河流域开展了富有成效的合作,通过()积极应对跨境犯罪、航行安全等问题。



\textbf{选项:}
\begin{itemize}[leftmargin=*]

  \item A. “澜湄快线”

  \item B. “澜湄合作机制”

  \item C. “南海生态走廊”

  \item D. 多边主义

\end{itemize}

\textbf{正确答案:}
B

\vspace{0.3em}\hrulefill\vspace{0.7em}

\subsection*{(三)合力打造“五大家园”的共同实践}

\subsubsection*{47. (单选题) \small QID: 19303020}

\textbf{题干:}
文化层面,中国-东盟命运共同体植根于多层次的文明交流,正是这种深层文化纽带,使双方合作超越了()的利益考量,具备了更为持久的生命力。



\textbf{选项:}
\begin{itemize}[leftmargin=*]

  \item A. 现实主义

  \item B. 多边主义

  \item C. 功利主义

  \item D. 社会主义

\end{itemize}

\textbf{正确答案:}
C

\vspace{0.3em}\hrulefill\vspace{0.7em}

\subsection*{(二)立足于传统文化}

\subsubsection*{49. (单选题) \small QID: 19303022}

\textbf{题干:}
构建周边命运共同体,延续了“天下为公”()的理想追求,是共建和平稳定周边秩序的方案选择,体现了中国共产党为人类进步事业不断奋斗的责任使命。



\textbf{选项:}
\begin{itemize}[leftmargin=*]

  \item A. “大同社会”

  \item B. “小康社会”

  \item C. “社会主义”

  \item D. “马克思主义”

\end{itemize}

\textbf{正确答案:}
A

\vspace{0.3em}\hrulefill\vspace{0.7em}

\subsection*{(四)强调原则性和灵活性相结合}

\subsubsection*{52. (多选题) \small QID: 19303025}

\textbf{题干:}
中国外交坚持用马克思主义唯物辩证法的世界观和方法论处理周边外交问题,坚持()和()有机统一,既敢于斗争又善于斗争,既有深邃洞见又有果断行动,使中国能够掌握外交主动,在坚定捍卫国家主权的同时,维护周边环境的和平稳定。



\textbf{选项:}
\begin{itemize}[leftmargin=*]

  \item A. 普遍联系

  \item B. 原则的坚定性

  \item C. 与时俱进

  \item D. 策略的灵活性

\end{itemize}

\textbf{正确答案:}
B | D

\vspace{0.3em}\hrulefill\vspace{0.7em}

\section*{破局与立新:“双碳”目标下的中国绿色发展路径及青年使命}
\hrulefill

\subsection*{(一)碳达峰 (Peak Carbon Dioxide Emissions)和碳中和 (Carbon Neutrality)}

\subsubsection*{7. (单选题) \small QID: 19303507}

\textbf{题干:}
推动“双碳”目标的纲领性文件是()。



\textbf{选项:}
\begin{itemize}[leftmargin=*]

  \item A. 《十四五规划》

  \item B. 《巴黎协定》

  \item C. 《京都议定书》

  \item D. 《哥本哈根协议》

\end{itemize}

\textbf{正确答案:}
A

\vspace{0.3em}\hrulefill\vspace{0.7em}

\subsection*{(二)温室气体 (Greenhouse Gases)}

\subsubsection*{26. (多选题) \small QID: 19303527}

\textbf{题干:}
2023年国内外发布的温室气体公报包括()。



\textbf{选项:}
\begin{itemize}[leftmargin=*]

  \item A. 《2022年全球温室气体公报》

  \item B. 《2022年中国温室气体公报》

  \item C. 《美国气候白皮书》

  \item D. 《欧盟减排计划》

\end{itemize}

\textbf{正确答案:}
A | B

\vspace{0.3em}\hrulefill\vspace{0.7em}

\subsection*{(二)温室气体 (Greenhouse Gases)}

\subsubsection*{36. (多选题) \small QID: 19303544}

\textbf{题干:}
中国应对气候变化的措施包括()。



\textbf{选项:}
\begin{itemize}[leftmargin=*]

  \item A. 发布温室气体公报

  \item B. 退出国际气候协议

  \item C. 启动自愿减排交易市场

  \item D. 建立排放因子数据库

\end{itemize}

\textbf{正确答案:}
A | C | D

\vspace{0.3em}\hrulefill\vspace{0.7em}

\subsection*{(三)绿色发展 (Green Development)}

\subsubsection*{53. (多选题) \small QID: 19303585}

\textbf{题干:}
绿色发展是关系我国发展全局的重要理念,是()的必然选择。



\textbf{选项:}
\begin{itemize}[leftmargin=*]

  \item A. 突破资源环境瓶颈制约

  \item B. 转变发展方式

  \item C. 实现可持续发展

  \item D. 高质量发展

\end{itemize}

\textbf{正确答案:}
A | B | C | D

\vspace{0.3em}\hrulefill\vspace{0.7em}

\subsection*{(三)绿色发展 (Green Development)}

\subsubsection*{71. (多选题) \small QID: 19303624}

\textbf{题干:}
建设绿色制造体系和服务体系,要积极推动()、()、()、()等新兴技术与绿色低碳产业深度融合。



\textbf{选项:}
\begin{itemize}[leftmargin=*]

  \item A. 互联网

  \item B. 大数据

  \item C. 人工智能

  \item D. 第五代移动通信(5G)

\end{itemize}

\textbf{正确答案:}
A | B | C | D

\vspace{0.3em}\hrulefill\vspace{0.7em}

\subsection*{(三)绿色发展 (Green Development)}

\subsubsection*{74. (单选题) \small QID: 19303627}

\textbf{题干:}
中国“双碳”目标与现代化建设的关系是()。



\textbf{选项:}
\begin{itemize}[leftmargin=*]

  \item A. 同步实现

  \item B. 相互冲突

  \item C. 先后顺序

  \item D. 无关

\end{itemize}

\textbf{正确答案:}
A

\vspace{0.3em}\hrulefill\vspace{0.7em}

\subsection*{概要}

\subsubsection*{78. (多选题) \small QID: 19303632}

\textbf{题干:}
“双碳”目标的战略意义包括()。



\textbf{选项:}
\begin{itemize}[leftmargin=*]

  \item A. 保障中华民族永续发展

  \item B. 构建人类命运共同体

  \item C. 扩大化石能源使用

  \item D. 优先发展重工业

\end{itemize}

\textbf{正确答案:}
A | B

\vspace{0.3em}\hrulefill\vspace{0.7em}

\subsection*{(一)“双碳”目标的提出是中国主动承担应对全球气候变化责任的大国担当}

\subsubsection*{94. (多选题) \small QID: 19303653}

\textbf{题干:}
中国气候行动的国际影响包括()。



\textbf{选项:}
\begin{itemize}[leftmargin=*]

  \item A. 提振全球信心

  \item B. 仅服务本国利益

  \item C. 免除他国责任

  \item D. 提供可再生能源方案

\end{itemize}

\textbf{正确答案:}
A | D

\vspace{0.3em}\hrulefill\vspace{0.7em}

\subsection*{(二)“双碳”目标是加快生态文明建设和实现高质量发展的重要抓手}

\subsubsection*{108. (多选题) \small QID: 19303667}

\textbf{题干:}
婺源县推行"河长制"的特点包括()。



\textbf{选项:}
\begin{itemize}[leftmargin=*]

  \item A. 政府主导

  \item B. 属地管理

  \item C. 分级负责

  \item D. 部门协作

\end{itemize}

\textbf{正确答案:}
A | B | C | D

\vspace{0.3em}\hrulefill\vspace{0.7em}

\subsection*{(三)贯彻新发展理念,推进创新驱动的绿色低碳高质量发展}

\subsubsection*{111. (单选题) \small QID: 19303670}

\textbf{题干:}
美国等发达国家试图通过()方式阻止中国在绿色低碳领域的发展。



\textbf{选项:}
\begin{itemize}[leftmargin=*]

  \item A. 科技脱钩

  \item B. 贸易壁垒

  \item C. 金融制裁

  \item D. 外交孤立

\end{itemize}

\textbf{正确答案:}
A

\vspace{0.3em}\hrulefill\vspace{0.7em}

\end{document}